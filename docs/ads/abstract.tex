%!TEX root = ../dokumentation.tex

\pagestyle{empty}

\iflang{de}{%
% Dieser deutsche Teil wird nur angezeigt, wenn die Sprache auf Deutsch eingestellt ist.
\renewcommand{\abstractname}{\langabstract} % Text für Überschrift

% \begin{otherlanguage}{english} % auskommentieren, wenn Abstract auf Deutsch sein soll
\begin{abstract}
The goal of this student research project is the development of an audio player for a Raspberry Pi in the programming language Go. The audio player enables the playback, pausing and stopping of audio files as well as the management of a playlist and the adjustment of the audio volume.
In order to realize this, a Raspberry Pi was provided and libraries for Go were used to decode an MP3 file, for example.
The result of this work is an extensive audio player for the Raspberry Pi with many functions, which can be used for many operational purposes, e.g. for the theatre group.
\end{abstract}
}



%Das Ziel dieser Studienarbeit ist die Entwicklung eines Audioplayers für einen Raspberry Pi in der Programmiersprache Go. Der Audioplayer ermöglicht das Abspielen, Pausieren und Stoppen von Audiodateien sowie das führen einer Playlist und das Anpassen der Wiedergabelautstärke.
%Um dies zu realisieren wurde auf Basis eines bereitgestellten Raspberry Pi entwickelt und für die Umsetzung Bibliotheken für Go verwendet, die z.B. das dekodieren einer MP3 Datei ermöglichen.
%Das Ergebnis dieser Arbeit ist ein umfangreicher Audioplayer für den Raspberry Pi mit vielen Funktionen, welcher für viele Einsatzzwecke wie z.B. für die Theatergruppe genutzt werden kann.