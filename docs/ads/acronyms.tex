%!TEX root = ../dokumentation.tex

\addchap{\langabkverz}
%nur verwendete Akronyme werden letztlich im Abkürzungsverzeichnis des Dokuments angezeigt
%Verwendung: 
%		\ac{Abk.}   --> fügt die Abkürzung ein, beim ersten Aufruf wird zusätzlich automatisch die ausgeschriebene Version davor eingefügt bzw. in einer Fußnote (hierfür muss in header.tex \usepackage[printonlyused,footnote]{acronym} stehen) dargestellt
%		\acs{Abk.}   -->  fügt die Abkürzung ein
%		\acf{Abk.}   --> fügt die Abkürzung UND die Erklärung ein
%		\acl{Abk.}   --> fügt nur die Erklärung ein
%		\acp{Abk.}  --> gibt Plural aus (angefügtes 's'); das zusätzliche 'p' funktioniert auch bei obigen Befehlen
%	siehe auch: http://golatex.de/wiki/%5Cacronym
%	
\begin{acronym}[YTMMM]
\setlength{\itemsep}{-\parsep}
\acro{ARM}{Acorn Risc Machine}
\acro{API}{Application Programming Interface}
\acro{APT}{Advanced Packaging Tool}
\acro{CI}{Continious Integration}
\acro{CPU}{Central Processing Unit}
\acro{CMD}{Command Prompt}
\acro{GB}{Gigabyte}
\acro{INI}{Initialisierungsdatei}
\acro{YAML}{YAML Ain’t Markup Language}
\acro{XML}{Extensible Markup Language}
\acro{GHz}{Gigahertz}
\acro{JSON}{JavaScript Object Notation}
\acro{GPIO}{General Purpose Input/Output}
\acro{HDMI}{High Definition Multimedia Interface}
\acro{HTML}{HyperText Markup Language}
\acro{NOOBS}{New Out of the Box Software}
\acro{RAM}{Random-Access Memory}
\acro{SD-Karten}{Secure Digital Memory Card}
\acro{SDLC}{Software Development Life Cycle}
\acro{Sudo}{Switch user do}
\acro{Tar}{Tape archiver}
\acro{USB}{Universal Serial Bus}
\acro{WLAN}{Wireless Local Area Network}
\end{acronym}
