%!TEX root = ../dokumentation.tex

\chapter{Aufgabenstellung}

Im Rahmen der Studienarbeit soll ein Musikplayer für den Raspberry Pi entworfen
und umgesetzt werden. Die Realisierung soll mit der Sprache Go erfolgen. Als
Grundlage können alle bereits bestehenden Musikplayer für den Raspberry Pi
dienen. Auch gibt es keine Einschränkungen, was den Einsatz von Bibliotheken
oder ähnlichem zur Musiksteuerung angeht. \hfill \break

Wichtig ist, dass der entstehende Musikplayer zukünftig von einem externen Tool aus
angesteuert werden soll. Deshalb muss der Player eine Schnittstelle bieten,
welche diesen Austausch unterstützt. Auch ist gefordert, dass die Musik im
Hintergrund abgespielt wird, um ein störungsfreies Arbeiten im Vordergrund
weiter zu ermöglichen. \hfill \break

An Funktionen soll der Fokus vor allem auf dem Starten und Stoppen von
Audiodateien. Erstrebenswert wäre dennoch die Umsetzung möglichst vieler der
folgenden Funktionen: 

\begin{itemize}
\item Starten,Stoppen und Pausieren der Wiedergabe der Audiodatei
\item Verändern der Ausgabelautstärke
\item Neustarten und überspringen von Audiodateien
\item Audiodateien in eine Playlist hinzufügen und Verwalten
\item Fade-In und Fade-Out Funktionalität ermöglichen
\end{itemize}

