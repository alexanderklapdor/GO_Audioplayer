%!TEX root = ../dokumentation.tex

\chapter{Aufgabenstellung}
Die Aufgabe im Rahmen der Studienarbeit umfasst die Entwicklung eines
Musikplayers für den Raspberry Pi. Die Realisierung dieses Players soll in der
Sprache Go erfolgen. Als Grundlage können alle bereits bestehenden Musikplayer
für den Raspberry Pi dienen. Auch gibt es keinerlei Einschränkungen was den
Einsatz von Bibliotheken oder ähnlichem bei der Entwicklung angeht. \hfill
\break

Wichtig ist, dass der entstehende Musikplayer zukünftig von einem externen Tool aus
angesteuert werden kann. Deshalb muss der Player eine Schnittstelle bieten,
welche diesen Austausch unterstützt. Auch ist gefordert, dass die Musik im
Hintergrund abgespielt wird, um ein störungsfreies Arbeiten im Vordergrund
parallel zu ermöglichen. \hfill \break
Der Fokus soll vor allem auf Funktionen wie dem Starten und Stoppen von
Audiodateien gesetzt werden. Erstrebenswert wäre weiterhin die Umsetzung
möglichst vieler der folgenden Funktionen: 

\begin{itemize}
\item Starten, Stoppen und Pausieren der Wiedergabe der Audiodatei
\item Verändern der Ausgabelautstärke
\item Neustarten und überspringen von Audiodateien
\item Audiodateien in eine Playlist hinzufügen und Verwalten
\item Fade-In und Fade-Out Funktionalität ermöglichen
\end{itemize}

