%!TEX root = ../dokumentation.tex

\chapter{Testung}

%title wird unter dem Bsp. abgedruckt
%caption wird im Verzeichnis abgedruckt
%label wird zum referenzieren benutzt, muss einzigartig sein.

\section{Continuous testing}
http://www.professionalqa.com/assets/images/continuous-testing.gif
\\
Hier einfach kurz erklären was wir da gemacht haben und was wir daraus für einen Vorteil gezogen haben

\section{Logging}
Als Logging bezeichnet man in der Informatik die automatische Erstellung eines Protokolls (englisch: log) von Softwareprozessen.
\\
Wofür und was wir für Vorteile daraus bekommen haben
\\
Ebenen erklären:
Info - Generally useful information to log (service start/stop, configuration assumptions, etc). Info I want to always have available but usually don't care about under normal circumstances. This is my out-of-the-box config level.
\\
Notice - Simply a statement that is non-actionable, use these to alert the user of something smaller and passive that you want the use to notice, such as an event that has happened like successful submit.
\\
Warn - Anything that can potentially cause application oddities, but for which I am automatically recovering. (Such as switching from a primary to backup server, retrying an operation, missing secondary data, etc.)
\\
Error - Any error which is fatal to the operation, but not the service or application (can't open a required file, missing data, etc.). These errors will force user (administrator, or direct user) intervention. These are usually reserved (in my apps) for incorrect connection strings, missing services, etc.
\\
Critical - Any error that is forcing a shutdown of the service or application to prevent data loss (or further data loss). I reserve these only for the most heinous errors and situations where there is guaranteed to have been data corruption or loss.



