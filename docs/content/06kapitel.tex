%!TEX root = ../dokumentation.tex

\chapter{Testung}
\begin{comment}
- Software testing during the software development
- Shift left testing
- quality assurance 
- detect defect as early as possible
- automatic testing
- goal of basic testing:
  - detect the obvious bugs that jump out immediately
  - otherwise: you get an expensive and unnexessary cycle of having to describe the problem of the devleoper,
  - who then has to repoduce it, debug it, solve it before trying again
- Basic steps for testing
- basic functionality testing
  - basic functions
- Code review
  - another pair of eyes
- static code analysis
  - anaylsis of your code through programs
  - coding standards
- unit testing
  - continous integration environment, unit should run on every commit
  - coverage goals
\end{comment}


Testen ist eines der Kernelemente bei der Entwicklung von Software. Dabei hat
sich die Vorgehensweise in den letzten Jahren stark verändert. Anstatt die
unterschiedlichen Phasen der Planung, Entwicklung und Testung strikt
voneinander abzutrennen, geht man aktuell immer mehr dazu über, die einzelnen
Phasen miteinander zu verknüpfen. \newline
Das \enquote{Shift left testing} ist ein Ansatz, welcher das Testen und
Entwickeln sehr stark miteinander verbindet. Das Ziel dabei ist es, Fehler
früher im Lebenszyklus zu finden und so das Beheben dieser einfacher zu
machen, was schlussendlicher zu einer signifikanten Kostensenkung führt. Denn
je später ein Fehler im Entwicklungszyklus entdeckt wird, desto mehr Kosten
verursacht er. \hfill \break


\section{Vorgehensweise}
Aufgrund der Überschaubarkeit des Projektes fällt die Entscheidung auf
paralleles Testen und Entwickeln durch Debugging. Dabei wird die Software
direkt auf dem Raspberry Pi entworfen, um umgesetze Features unmittelbar in
einer anwendungsnahen Umgebung im Betrieb sehen zu können. Dies bietet den
großen Vorteil, dass Fehler schneller auffallen und entsprechende
Gegenmaßnahmen direkt gestartet werden können. \newline
Des Weiteren wird bei der Entwicklung noch das \textit{Code Reviewing}
eingesetzt. Dabei schauen mehrere Personen über den Programmcode des
Entwicklers, um mögliche Schwachstellen frühzeitig dem Entwickler aufzuzeigen
zu können. 

http://www.professionalqa.com/assets/images/continuous-testing.gif
\\

\section{Logging}
Um beim Debuggen schneller Fehler zu erkennnen und um besser nachvollziehen zu
können, was das Programm aktuell macht, haben wir auf den Eisatz eines
selbstgeschriebenen \textit{Loggers} gesetzt. Dieser protokolliert wichtige
Ereignisse von Softwareprozessen, und gibt diese dem Entwickler sichtbar aus.
Diese Ausgabe kann je nach Einstellung sowohl in eine Datei erfolgen, als auch
direkt in den \textit{Stdout}. Um gewöhnliche Ausgaben von Warnungen oder
Fehlern abtrennen zu können, wird beim Logging mit verschiedenen Logging-Ebenen
gearbeitet, welche mit den Nachrichten ausgegeben werden. Nachfolgend werden
die verschiedenen Ebenen kurz erörtert:

\begin{description}
\item[Info] \hfil \\
Unter \textit{Info} werden allgemeingültige Informationen verstanden, welche zwar
interessant sind, aber unter normalen Umständen zu vernachlässigen sind (z.B.
Starten/Stoppen eines Services, Änderung von Konfigurationen, etc.).
\item[Notice] \hfil \\ 
\textit{Notice} ist ein einfaches Statement, welches von einer nicht
handlungsgesteuerten Aktion ausgelöst wird. Beispiele dafür wären kleine
passive Benachrichtigungen an den Benutzer, wenn beispielsweise einr Aktion
erfolgreich abgeschlossen wurde.
\item[Warn] \hfil \\ 
Mit \textit{Warn} werden ungewöhnliche Aktionen gekennzeichnet, welche die
Lauffähigkeit des Programmes einschränken könnten, allerdings
automatisch durchs Programm bemerkt und behoben wurden (z.B. beim Wechseln vom
Primärserver auf einen Backup Server, erneutes Versuch einer Operation, etc.)
\item[Error] \hfil \\ 
Unter einem \textit{Error} wird das Fehlschlagen von Operationen verstanden,
welche nicht gleich die Applikation oder den Service lahmlegen (z.B. Fehlschlag
beim Öffnen einer wichtigen Datei, Fehlen von wichtigen Daten, etc.). Diese
Fehler zwingen den Nutzer zum manuellen Eingreifen in das Geschehen.
\item[Critical] \hfil \\ 
Mit \textit{Critical} werden schwerwiegende Fehler verstanden, welche
zwangsweise zu dem Verlust von Daten durch das Beenden des Services
oder der Applikation führt.
\end{description}


