%!TEX root = ../dokumentation.tex

\chapter{Analyse}
Bevor es an die tatsächliche Realisierung des Projektes geht, müssen zunächst
bestehende Bibliotheken und Technologien auf ihre Nutzbarkeit im Bezug auf das
Projekt analysiert werden. 

%comment: evtl nen Raspberry Editionen verleich hinzufügen

\section{Einarbeitung}
Bevor es an die Realisierung der Aufgabe geht, steht zunächst das
Vertrautmachen mit den unterschiedlichen Technologien im Vordergrund. Für die
Entwicklung wurde eine Raspberry Pi 3 Modell B bereitgestellt. Da das zu
entwickelnde Programm später auf dem gleichen Modell laufen wird, wird auf
einen Vergleich der veschiedenen Modelle verzichtet. Bevor der Raspberry
benutzt werden kann, muss zunächst ein Betriebssystem installiert werden.
\hfill \break

Für den Raspberry Pi gibt es verschiedene Distributionen, die auf verschiedene
Anwendungsszenarien ausgelegt sind. Nachfolgend sind ein paar der verfügbaren
Betriebssystem aufgelistet: 
\begin{description}
\item[Raspbian] \hfill \\ 
    Raspbian ist eines der ältesten und meistverbreitetsten Betriebssysteme für
    den Raspberry Pi. 
\item[Pi MusicBox] \hfill \\ 
    Bei der Pi MusicBox handelt es sich um eine Distribution, welche auf das
    Abspielen von Musik ausgelegt ist. Es unterstützt alle bewährten
    Streaming-Dienste oder auch das Streamen von Musik aus dem Netzwerk.
\item[RetroPie] \hfill \\
    Der RetroPie ist eine Distribution, welche das Emulieren von klassischen
    Spielekonsolen ermöglicht.
\end{description}

Im Rahmen dieser Arbeit geht es zwar um das Abspielen von Musik, allerdings
soll dafür ein eigenständiges Tool entwickelt werden. Da Raspbian das
offizielle Betriebssystem für den Raspberry Pi ist, es am verbreitetsten ist
und es dafür die meisten Pakete gibt, wird sich im Bezug auf diese Arbeit für
das Betriebssystem Raspbian entschieden. \hfill \break

Beim Installieren von Raspbian kann man auf einen :
Um Raspbian auf dem Raspberry Pi zu installieren gibt es zwei Möglichkeiten:
\ac{NOOBS} oder Raspbian verwenden. \ac{NOOBS} ist ein Operating System
Installer, welche durch seine Einfachheit besonders anfängerfreundlich ist.
Standardmäßig kommt er mit einer Installation von Raspbian und LibeELEC, kann
aber zusätzlich noch weitere Betriebssysteme laden. Wenn man nur Raspbian
verwenden will, kann man sich die Installation von \ac{NOOBS} sparen und direkt
Raspbian nutzen. Da wir in unserem Projekt auch nur Raspbian verwenden werden,
haben wir uns für die zweite Variante entschieden.


\section{Abspielen der Musik}
Um bei der Umsetzung nicht komplett von Null anfangen zu müssen, geht es
zunächst an die Analyse einer geeigneten Audio Bibliothek, um das Abspielen von
Audiodateien zu ermöglichen. Dabei ist es wichtig, dass die Biblitohek frei zu
verwenden, gut dokumentiert ist und eine Schnittstelle für Go besitzt.  Im
ersten Schritt der Analyse geht es an die Entscheidung einer geeigneter Audio
Bibliothek, um das Abspielen von Audiodateieen zu ermöglichen. Wichtig dabei
ist, dass die Bibliothek keine Einschränkungen bei der Benutzung hat, gut
dokumentiert ist und dass die Bibliothek bestenfalls eine bestehende
Schnittstelle für Go besitzt.
\subsection{Bibliotheken}
\paragraph{PortAudio}
PortAudio ist eine Open Source Bibliothek, welche das Abspielen und Aufnehmen
von Audiodateien ermöglicht. Sie bietet eine plattformübergreifende Lösung,
wodurch sie auf den gängisten Betriebssystemen problemlos läuft. Einer der
Gründ dafür, dass sie plattformübergreifend funktioniert, ist der, dass
PortAudio in der Programmiersprache C geschrieben wurde. \hfill \break

Die Kommunikation mit PortAudio verläuft über eine \ac{API} Schnittstelle.
Diese entnimmt den Datenstrom entgegen, welche abgespielt werden soll. Während
des Abspiel- oder Aufnahmprozesses von Audiodateien benutzt PortAudio entweder
eine Callback-Funktion oder einen blockierendes Read/Write Interface.
\href{http://www.portaudio.com/}{Portaudio}

\paragraph{libsoundio}
libsoundio ist genau wie PortAudio eine Open Source Bibliothek, die eine
Schnittstelle zur Ein- und Ausgabe von Audiostreams ermöglicht. Sie zeichnet
sich dadurch aus, dass sie plattformübergreifend und sehr gut dokumentiert ist.
libsoundio stellt eine leichtgewichtige Abstraktion über verchiedenste
Soundtreiber dar.

\subsection{Entscheidung}
Nach unseren Recherchen können wir keine großartigen Unterschiede zwischen den
beiden Grundkonzepten der Audio-Bibliotheken feststellen. Ein Vorteil von
libsoundio ist ganz klar, dass sie eine ausgesprochen umfangreiche
Dokumentation besitzt, in der der Entwickler einen großen Fokus auf
Vollständigkeit gelegt hat. PortAudio hingegen ist insgesamt weiter verbreitet.
Da es in Go eine Bibliothek gibt, welche im Hintergrund auf die Schnittstelle
von PortAudio zurückgreift, haben wir uns im Rahmen der Arbeit für die
Verwendung von PortAudio entschieden.

\section{MP3 Decoder}
Bei MP3-Dateien handelt es sich um komprimierte Audio-Dateien. Im Vergleich zu
anderen Audioformaten, setzt MP3 auf das Psychoakustik-Modell. Dieses besagt,
dass Menschen nur in einer Frequenz von 20 Hz bis 20 kHz hören kann. MP3
filtert deshalb alle anderen für den Menschen nicht hörbaren Bereich raus, und
erreicht so eine Speicherersparnis um mehr als den Faktor 10 im Vergleich zum
Rohformat. \hfill \break

Da wir im Rahmen der Arbeit auch die Möglichkeit zum Abspielen von MP3-Dateien
bieten wollen, muss bei der Implementierung auch auf einen MP3 Decoder zur
Dekodierung der MP3-Dateien zurückgegriffen werden. Entschieden haben wir uns
zu diesem Zweck für die Go-Bibliothek \textit{go-mpg123}. Diese greift im
Hintergrund auf die Bibliothek libmpg123. Aktuell befindet sich die Bibliothek
noch in der Entwicklung, allerdings bietet sie bereits die geforderte
Funktionalität zur Dekodierung an.
\href{https://github.com/bobertlo/go-mpg123}{MP3 Decoder}

\section{MP3 Tag Info Reader}
MP3 Dateien können neben den reinen Audiodaten auch noch zusätzliche
Informationen speichern (Metadaten). Dafür verwendet MP3 ein Format namens
\textit{ID3} (identity an MP3), welches einzelne Information in
sogenannten \textit{ID3-Tags} speichert. In diesen Tags werden meist
Informationen wie Interpreten, Titel oder auch Albumname gespeichert. Da diese
Informationen in einem Musikplayer optimalerweise angezeigt werden sollten,
muss nach einer Möglichkeit zum Auslesen dieser Tags gesucht werden. \hfill
\break
Aktuell unterstützt \textit{go-mp3} noch nicht das Auslesen der ID3 Tags.
Deswegen muss nach einer anderen Bibliothek zur Auslesung dieser Daten gesucht
werden. Die Entscheidung fiel dabei auf die Go-Bibliothek \textit{ID3 Decoder},
da diese einfach zu verwenden ist und die meist-verbreitetste Go-Bibliothek
dafür ist.
\href{https://github.com/mikkyang/id3-go}{ID3 Decoder}
