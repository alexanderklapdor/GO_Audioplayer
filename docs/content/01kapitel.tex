%!TEX root = ../dokumentation.tex

\chapter{Einleitung}
Aktuell kann der Trend beobachtet werden, dass etablierte Applikationen oder
Programme dazu übergehen, zusätzliche Funktionalitäten zu ihren Grundfeatures
hinzuzufügen. So umfasst \textbf{beispielsweise} die beliebteste Handyapplikation in China namens
\textit{WeChat} weitaus mehr Funktionen als nur zum einfachen
Nachrichtenaustausch, wofür sie anfänglich entwickelt wurde. Mittlerweile
bietet die Applikation sogar Möglichkeiten zur Navigation, zum
Einkaufen von Artikeln, zum Managen von Flügen oder auch zur Verwaltung von
Zahlungen \autocite{was_ist_wechat?_2019}. Auch \textit{WhatsApp} hat begonnen ihren
Fokus auf zusätzliche Funktionen zu erweitern. So stellt WhatsApp mittlerweile
Möglichkeiten für Internetanrufe und Videochat zur Verfügung, und arbeitet
aktuell an der Einführung eines Bezahl-Features \autocite{shead_2019}. \hfill
\break


Diese \glqq Allround-Lösungen\grqq \space ermöglichen es zwar die Anzahl der
benötigten Applikationen zu reduzieren, schränken aber die Individualität der
Nutzer ein, da die einzelnen Features oft fest implementiert sind und nicht
einfach aus der App oder dem Programm entfernt werden können. Dies führt
oftmals zu einer unnötigen Speicherverschwendung. Des Weiteren können diese
Softwarepakete meist den Einsatz von Spezialprogrammen nicht ersetzen, da diese
nur die Grundfunktionalitäten enthalten. \hfill \break


Auch durch die zuvor genannten Punkte, hat sich in den letzen Jahren der Trend
von \acp{API} weiter fortgesetzt. Immer mehr Entwickler setzen darauf
Schnittstelle anzubieten, die den einfachen Austausch von Informationen mit den
Programmen ermöglichen. Dies vereinfacht das Verknüpfen von verschiedenen
Programmen ungemein, und bietet den großen Vorteil, dass man als Entwickler auf
etablierte Speziallösungen zurückgreifen kann, und nicht alles von Grund auf
neuentwickeln muss. So entstehen anstatt riesiger in sich geschlossener Programme
dynamische Softwarepakete, welche von unterschiedlichsten Entwicklern weiter
entwickelt werden \autocite{babati_2018} \autocite{sandoval_2018}. \hfill
\break


Im Rahmen dieser Arbeit soll keinesfalls eine perfekt ausgearbeitete
Komplettlösung erarbeitet werden, sondern ganz nach der ersten Regel der
Unix-Philsophie:
\begin{quote}
\glqq Schreibe Computerprogramme so, dass sie nur eine Aufgabe erledigen und diese gut machen\grqq \autocite{gancarz_1997}
\end{quote}
lediglich ein Programm, welches die Grundfunktionalitäten eines Musikplayers erfüllt.
\hfill \break


Unix Philosophie
\begin{itemize}
\item Schreibe Computerprogramme so, dass sie nur eine Aufgabe erlediegen und
diese gut machen.
\item Schreibe Programme so, dass sie zusammenarbeiten
\item Schreibe Programme so, dass sie Textströme verarbeiten, denn das ist eine
universelle Schnittstelle.
\end{itemize}


\section{Ist-Zustand}
Zur Technikverwaltung einer Theatergruppe wird aktuell bereits ein Raspberry
Pi eingesetzt, um während des Spiels die Steuerung des Lichts übernehmen zu
können. So können die Lichter von einer zentralen Stelle aus gesteuert werden,
und automatisierte Lichtabläufe erstellt werden. Während Aufführungen oder
Proben werden neben Lichteffekte auch noch Musik zur Untermalung des
Schauspiels eingesetzt. Die Steuerung der Musik wird aktuell noch manuell von
einem Musikplayer aus vorgenommen. Dieser soll in Zukunft abgelöst werden. \\
\\
\textbf{ Stimmt alles so , nur das für das Musik abspielen eben ein Kabel (Von der Bühne bis zum ende des Raumes) gelegt werden muss, da sich da die Technik zum steuern/abspielen der Musik befindet. Dieses Kabellegen soll in Zukunft abgelöst werden, sodass eben alles zentral gesteuert werden kann(Musik und Licht)}

\section{Soll-Zustand}
Zur Vereinfachung der Steuerung einer Theatergruppe sollen zukünftig alle
relevante Elemente von einem zentralen Ort aus gesteuert werden können. Dies
soll die Arbeit der Personen hinter der Technik erleichtern\textbf{(Keine Kabel mehr legen)}, und das Auftauchen
von Fehlern reduzieren. Des Weiteren soll es möglich sein, automatisierte
Abläufe \textbf{aus der Mischung von Musik und Licht} zu erstellen, sodass die Technikleute nur noch eine passive
Überwachungsrolle einnehmen müssen, und ein manuelles Eingreifen nur noch im
Notfall vonnöten ist. \textbf{Weiß nicht ob das mit dem nur noch eingreifen wirklich sein Ziel ist...kann aber drinnen bleiben würde ich sagen...glaube die wollen halt einfach alles vom pc aus zentral steuern können}

\begin{itemize}
\item Theatergrupppe
\item RasberryPi
\item Musik abspielen
\item Lichtsteuerung bereits über RaspberryPi
\item Programmablauf integrieren
\item Automatisierung
\end{itemize}

\section{Aufgabe}
Im Rahmen dieser Arbeit soll die Verwaltung der Technik einer Theatergruppe
vereinfacht werden. Dabei soll im Speziellen eine Schnittstelle angelegt
werden, welche das Abspielen von Musik von einer externen Software aus
ermöglicht. Das dabei entstehende Programm soll auf einem Raspberry Pi laufen,
um zukünftig die komplette Steuerung der Technik von diesem Gerät aus zu
ermöglichen.

