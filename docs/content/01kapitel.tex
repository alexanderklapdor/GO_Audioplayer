%!TEX root = ../dokumentation.tex

\chapter{Einleitung}
Aktuell kann der Trend beobachtet werden, dass etablierte Applikationen oder
Programme dazu übergehen, zusätzliche Funktionalitäten zu ihren Grundfeatures
hinzuzufügen. So umfasst die beliebteste Handyapplikation in China namens
\textit{WeChat} weitaus mehr Funktionen als nur zum einfachen
Nachrichtenaustausch, wofür sie anfänglich entwickelt wurde. Mittlerweile
bietet die Applikation sogar Möglichkeiten zur Navigation, zum
Einkaufen von Artikeln, zum Managen von Flügen oder auch zur Verwaltung von
Zahlungen. Auch WhatsApp hat begonnen ihren Fokus auf zusätzliche Funktionen zu
erweitern. So stellt WhatsApp mittlerweile Möglichkeiten für Internetanrufe und
Videochat zur Verfügung, und arbeitet aktuell an der Einführung eines
Bezahl-Features. \hfill \break


Diese \glqq Allround-Lösungen\grqq \space ermöglichen es zwar die Anzahl der
benötigten Applikationen zu reduzieren, schränken aber die Individualität der
Nutzer ein, da die einzelnen Features oft fest implementiert sind und nicht
einfach aus der App oder dem Programm entfernt werden können. Dies führt
oftmals zu einer unnötigen Speicherverschwendung. Des Weiteren können diese
Softwarepakete meist den Einsatz von Spezialprogrammen nicht ersetzen, da diese
nur die Grundfunktionalitäten enthalten. \hfill \break


Auch durch die zuvor genannten Punkte, hat sich in den letzen Jahren der Trend
von \acp{API} weiter fortgesetzt. Immer mehr Entwickler setzen darauf
Schnittstelle anzubieten, die den einfachen Austausch von Informationen mit den
Programmen ermöglichen. Dies vereinfacht das Verknüpfen von verschiedenen
Programmen ungemein, und bietet den großen Vorteil, dass man als Entwickler auf
etablierte Speziallösungen zurückgreifen kann, und nicht alles von Grund auf
neuentwickeln muss. So entstehen anstatt riesiger in sich geschlossener Programme
dynamische Softwarepakete, welche von unterschiedlichsten Entwicklern weiter
entwickelt werden. \hfill \break


Im Rahmen dieser Arbeit soll keinesfalls eine perfekt ausgearbeitete
Komplettlösung erarbeitet werden, sondern ganz nach der ersten Regel der
Unix-Philsophie: (\glqq Schreibe Computerprogramme so, dass sie nur eine
Aufgabe erledigen und diese gut machen\grqq), lediglich ein Programm, welches
die Grundfunktionalitäten eines Musikplayers erfüllt. \hfill \break


Unix Philosophie
\begin{itemize}
\item Schreibe Computerprogramme so, dass sie nur eine Aufgabe erlediegen und
diese gut machen.
\item Schreibe Programme so, dass sie zusammenarbeiten
\item Schreibe Programme so, dass sie Textströme verarbeiten, denn das ist eine
universelle Schnittstelle.
\end{itemize}


\section{Ist-Zustand}
Im Rahmen der Studienarbeit soll die Verwaltung der Technik einer
Theatergrupppe vereinfacht werden. Aktuell wird bereits ein Raspberry Pi
verwendet, um während des Spiels die Steuerung des Lichts übernehmen
zu können. So können die Lichter von einer zentralen Stelle aus gesteuert
werden, und automatisierte Lichtabläufe erstellt werden. Während Aufführungen
oder Proben werden neben Lichteffekte auch noch Musik zur Untermalung verwendet
zur Untermalung verwendet. Die Steuerung der Musik wird aktuell noch manuell
von einem Musikplayer aus vorgenommen. Dieser soll in Zukunft abgelöst werden.

\section{Soll-Zustand}
Zuküntig soll es eine zentrale Stelle für die Steuerung der Musik und des
Lichts geben. Dafür soll eine Schnittstelle für die Musik entworfen werden,
sodass diese auch vom Raspberry Pi aus gesteuert werden kann. Die bietet den
Vorteil, dass es einen zentralen Ort für die Steuerung gibt. Des Weiteren
können so automatisierte Programmabläufe erstellt werden, welche ein manuelles
Eingreifen obsoloet machen.

\begin{itemize}
\item Theatergrupppe
\item RasberryPi
\item Musik abspielen
\item Lichtsteuerung bereits über RaspberryPi
\item Programmablauf integrieren
\item Automatisierung
\end{itemize}

\section{Aufgabe}
Im Rahmen dieser Studienarbeit soll ein Programm entwickelt werden, welches
die grundsätzlichen Funktionalitäten eines Musikplayers auf einem RaspberryPi
ermöglicht. Dieses Programm soll einfach gehalten werden und eine Schnittstelle
bieten, um von einem einem bestehenden externen Programm aus aufgerufen werden
zu können. 
