%!TEX root = ../dokumentation.tex

\chapter{Einleitung}
Aktuell kann der Trend beobachtet werden, dass etablierte Applikationen oder
Programme dazu übergehen, zusätzliche Funktionalitäten zu ihren Grundfeatures
hinzuzufügen. So umfasst beispielsweise die beliebteste Handyapplikation in China namens
\textit{WeChat} weitaus mehr Funktionen als nur zum einfachen
Nachrichtenaustausch, wofür sie anfänglich entwickelt wurde. Mittlerweile
bietet die Applikation sogar Möglichkeiten zur Navigation, zum
Einkaufen von Artikeln, zum Managen von Flügen oder auch zur Verwaltung von
Zahlungen \autocite{was_ist_wechat?_2019}. Auch \textit{WhatsApp} hat begonnen ihren
Fokus auf zusätzliche Funktionen zu erweitern. So stellt WhatsApp mittlerweile
Möglichkeiten für Internetanrufe und Videochat zur Verfügung, und arbeitet
aktuell an der Einführung eines Bezahl-Features \autocite{shead_2019}. \hfill


Diese \enquote{Allround-Lösungen} \space ermöglichen es zwar die Anzahl der
benötigten Applikationen zu reduzieren, schränken aber die Individualität der
Nutzer ein, da die einzelnen Features oft fest implementiert sind und nicht
einfach aus der App oder dem Programm entfernt werden können. Dies führt
oftmals zu einer unnötigen Speicherverschwendung. Des Weiteren können diese
Softwarepakete meist den Einsatz von Spezialprogrammen nicht ersetzen, da diese
nur die Grundfunktionalitäten enthalten. \hfill


Durch die zuvor genannten Punkte, hat sich auch in den letzten Jahren der Trend
von \acp{API} weiter fortgesetzt. Immer mehr Entwickler setzen darauf
Schnittstelle anzubieten, die den einfachen Austausch von Informationen mit den
Programmen ermöglichen. Dies vereinfacht das Verknüpfen von verschiedenen
Programmen ungemein, und bietet den großen Vorteil, dass man als Entwickler auf
etablierte Speziallösungen zurückgreifen kann, und nicht alles von Grund auf
neu entwickeln muss. So entstehen anstatt riesiger in sich geschlossener Programme
dynamische Softwarepakete, welche von unterschiedlichsten Entwicklern weiter
entwickelt werden \autocite{babati_2018} \autocite{sandoval_2018}. \hfill \break

Im Rahmen dieser Arbeit soll keinesfalls eine perfekt ausgearbeitete
Komplettlösung erarbeitet werden, sondern ganz nach der ersten Regel der
Unix-Philosophie:
\begin{quote}
\glqq Schreibe Computerprogramme so, dass sie nur eine Aufgabe erledigen und diese gut machen\grqq \autocite{gancarz_1997}
\end{quote}
lediglich ein Programm, welches die Grundfunktionalitäten eines Musikplayers erfüllt.


%Unix Philosophie
%\begin{itemize}
%\item Schreibe Computerprogramme so, dass sie nur eine Aufgabe erlediegen und
%5diese gut machen.
%\item Schreibe Programme so, dass sie zusammenarbeiten
%\item Schreibe Programme so, dass sie Textströme verarbeiten, denn das ist eine
%universelle Schnittstelle.
%\end{itemize}


\section{Ist-Zustand}
Zur Technikverwaltung einer Theatergruppe wird aktuell bereits ein Raspberry
Pi eingesetzt, um während des Spiels die Steuerung des Lichts übernehmen zu
können. So können die Lichter von einer zentralen Stelle aus gesteuert werden,
und automatisierte Lichtabläufe erstellt werden. Während Aufführungen oder
Proben wird neben Lichteffekte auch noch Musik zur Untermalung des
Schauspiels eingesetzt. Die Steuerung der Musik wird aktuell noch manuell von
einer Musikanlage aus vorgenommen. Diese Anlage befindet sich am Ende des
Raumes, weshalb zur Steuerung dieser immer ein Kabel quer durch den ganzen
Raum gelegt werden muss.

\section{Problemstellung}
Das Problem an der bestehenden Lösung ist zunächst, dass das manuelle
Kabellegen jedes Mal einen ungewollten Aufwand darstellt, und auch zu einer
weiteren Fehlerquelle wird, da es durch Personen beispielsweise beschädigt
werden kann. Des Weiteren gibt es aktuell noch keine Möglichkeit, automatisierte
Abläufe mit Licht und Musik zu erstellen. Dies führt dazu, dass wiederholende
Abläufe manuell gesteuert werden muss, was immer ein erhöhtes Fehlerpotenzial
darstellt. Dies sollte zukünftig vermieden werden.


\section{Soll-Zustand}
Zur Vereinfachung der Steuerung einer Theatergruppe sollen zukünftig alle
relevante Elemente von einem zentralen Ort aus gesteuert werden können. Dies
soll die Arbeit der Personen hinter der Technik erleichtern, und das Auftauchen
von Fehlern reduzieren. Auch der Aufbauaufwand wird verringert, da ein
Kabellegen nicht mehr nötig ist. Des Weiteren soll es möglich sein,
automatisierte Abläufe mit Musik und Licht zu erstellen, sodass die
Technikleute nur noch eine passive Überwachungsrolle einnehmen müssen, und ein
manuelles Eingreifen nur noch im
Notfall vonnöten ist. 


\section{Aufgabe}
Im Rahmen dieser Arbeit soll die Verwaltung der Technik einer Theatergruppe
vereinfacht werden. Dabei soll im Speziellen eine Schnittstelle angelegt
werden, welche das Abspielen von Musik von einer externen Software aus
ermöglicht. Das dabei entstehende Programm soll auf einem Raspberry Pi laufen,
um zukünftig die komplette Steuerung der Technik von diesem Gerät aus zu
ermöglichen.

