%!TEX root = ../dokumentation.tex

\chapter{Einleitung}
Aktuell kann der Trend beobachtet werden, dass etablierte Applikationen oder
Programme dazu übergehen zusätzliche Funktionalitäten zu ihren Grundfeatures
hinzuzufügen. So umfasst beispielsweise die beliebteste Handyapplikation in China namens
\textit{WeChat} weitaus mehr Funktionen als nur zum einfachen
Nachrichtenaustausch, wofür sie anfänglich entwickelt wurde. Mittlerweile
bietet die Applikation sogar Möglichkeiten zur Navigation, zum
Einkaufen von Artikeln, zum Managen von Flügen oder auch zur Verwaltung von
Zahlungen \autocite{was_ist_wechat?_2019}. Auch \textit{WhatsApp} hat begonnen ihren
Fokus um weitere Funktionen zu erweitern. So stellt WhatsApp inzwischen
Möglichkeiten für Internetanrufe und Videochat zur Verfügung, und arbeitet
aktuell an der Einführung eines Bezahl-Features \autocite{shead_2019}. \hfill


Diese \enquote{Allround-Lösungen} ermöglichen es zwar die Anzahl der
benötigten Applikationen zu reduzieren, schränken aber die Individualität der
Nutzer ein, da die einzelnen Features oft fest implementiert sind und nicht
einfach aus der App oder dem Programm entfernt werden können. Dies führt
oftmals zu einer unnötigen Speicherverschwendung. Des Weiteren können diese
Softwarepakete meist den Einsatz von Spezialprogrammen nicht ersetzen, da diese
nur die Grundfunktionalitäten enthalten. \hfill


Durch die zuvor genannten Punkte, hat sich auch in den letzten Jahren der Trend
von \acp{API} weiter fortgesetzt. Immer mehr Entwickler setzen darauf
Schnittstelle anzubieten, die den einfachen Austausch von Informationen zwischen verschiedenen
Programmen ermöglichen. Dies vereinfacht das Verknüpfen von unabhängigen
Programmen ungemein, und bietet den großen Vorteil, dass man als Entwickler auf
etablierte Speziallösungen zurückgreifen kann, und nicht alles von Grund auf
neu entwickeln muss. Dadurch entstehen anstatt riesiger in sich geschlossener Programme
dynamische Softwarepakete, welche von unterschiedlichsten Entwicklern weiter
entwickelt werden \autocite{babati_2018} \autocite{sandoval_2018}. \hfill \break

Im Rahmen dieser Arbeit soll keinesfalls eine perfekt ausgearbeitete
Komplettlösung erarbeitet werden, sondern, ganz nach der ersten Regel der
Unix-Philosophie \enquote{Schreibe Computerprogramme so, dass sie nur eine
Aufgabe erledigen und diese gut machen} \autocite{gancarz_1997}, lediglich ein
Programm, welches die Grundfunktionalitäten eines Musikplayers erfüllt.


%Unix Philosophie
%\begin{itemize}
%\item Schreibe Computerprogramme so, dass sie nur eine Aufgabe erlediegen und
%5diese gut machen.
%\item Schreibe Programme so, dass sie zusammenarbeiten
%\item Schreibe Programme so, dass sie Textströme verarbeiten, denn das ist eine
%universelle Schnittstelle.
%\end{itemize}


\section{Ist-Zustand}
Zur Technikverwaltung einer Theatergruppe wird aktuell ein Raspberry
Pi eingesetzt, um während des Spiels die Steuerung des Lichts zu übernehmen. Dadurch
können die Lichter von einer zentralen Stelle aus gesteuert werden,
und automatisierte Lichtabläufe erstellt werden. Während Aufführungen oder
Proben wird neben Lichteffekte auch noch Musik zur Untermalung des
Schauspiels eingesetzt. Die Steuerung dieser Musik wird aktuell noch manuell von
einer Musikanlage aus vorgenommen. Diese Anlage befindet sich am Ende des
Raumes, weshalb zur Steuerung dieser immer ein Kabel quer durch den ganzen
Raum gelegt werden muss.

\section{Problemstellung}
Die Problematik hinter der bestehenden Lösung ist, dass das manuelle Kabelverlegen
ein ungewollten Mehraufwand darstellt, welcher jedes Mal wieder von neuem
vollzogen werden muss. Des Weiteren stellt das Kabel eine neue mögliche
Fehlerquelle dar, da es durch die Installation quer durch den Raum von
Teilnehmern ungewollt beschädigt werden könnte.
Außerdem gibt es aktuell noch keine Möglichkeit, automatisierte
Abläufe zu erstellen, welche sowohl Licht als auch Musik umfassen. Dadurch
müssen wiederholende Abläufe manuell durch das Technikpersonal gesteuert
werden, was auch eine potentielle Fehlerquelle ist. Dies soll zukünftig
vermieden werden.


\section{Soll-Zustand}
Um die Steuerung der Technik einer Theatergruppe zu vereinfachen, sollen
zukünftig alle relevanten Elemente von einem zentralen Ort aus gesteuert
werden. Dadurch soll der Aufwand der Personen hinter der Technik wesentlich
verringert werden, und das Auftauchen durch Menschen verursachte Fehler
reduziert werden. Zusätzlich soll der Installationsaufwand verringert werden,
indem das Kabellegen für die Musikboxen obsolet gemacht wird.  Des Weiteren
soll es möglich sein, automatisierte Abläufe mit Musik und Licht zu erstellen,
sodass die Technikleute nur noch eine passive Überwachungsrolle einnehmen
müssen, und ein manuelles Eingreifen nur noch im Notfall vonnöten ist. 


\section{Aufgabe}
Die Aufgabe der Studienarbeit ist es, die Verwaltung der Technik durch die
Entwicklung eines Musikplayers zu vereinfachen. Dabei soll im Speziellen eine
Schnittstelle entwickelt werden, welche das Abspielen von Musik von einer
externen Software aus ermöglicht. Das dabei entstehende Programm soll auf einem
Raspberry Pi laufen, um zukünftig die komplette Steuerung der Technik von
diesem Gerät aus zu ermöglichen. Der große Vorteil dahinter ist, dass der
Raspberry Pi übers Netzwerk angesteuert werden kann, wodurch das lästige
Kabellegen nicht mehr nötig ist.

